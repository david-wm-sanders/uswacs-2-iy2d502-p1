\textcolor{deep-gray}{\textit{Notes to the reader: in order to align the content of this report with that of the presentation given to Chris Roberts, I have chosen to split this report into two parts. The first part of the report is an executive summary which details RFID technologies in access control contexts (the task for which they are most deployed in organisations). It was written with the intent that it should be possible to provide the executive summary as a piece of supplementary material to audiences viewing tangible demonstrations of the threat detailed. The second part of the report is dedicated to exploring the practical, legal, and ethical issues with regard to demonstrating these threats and tools -- it is directed at and written for those performing demonstrations.}}

\section{RFID Access Control Threats}
RFID \textit{(Radio-frequency identification)} technologies, of which NFC \textit{(Near-field communication)} is a specific implementation, allow for wireless \textit{(contactless)} communications between devices \textit{(such as access card readers and mobile phones)} and objects with embedded or attached RFID tags \textit{(such as access cards)}. RFID technologies are being used to carry out tasks such as ID badging, access control, personnel tracking, inventory management, asset tracking, counterfeit prevention, supply chain management, and contactless payment. Although RFID deployments can present considerable advantages from a commercial and financial point of view, they can also introduce new security concerns.\\

\noindent In the context of most businesses, a primary use of RFID is for ID badging and access control \textit{(where the ID badge tag also acts as the access control authenticator)}. Even this limited subset of use cases presents many opportunities for criminal or hostile state actors to attempt to interfere with the intended electronic operation of the RFID system. As RFID access control systems continue to become more widespread and bigger in deployment scale, the attack surface and attraction to hostile actors will continue to increase.\\

\noindent The major threats to such systems are the loss of confidential \textit{(and possibly personal)} information stored on tags to hostile actors by unauthorised tag reading and cloning or spoofing tags by hostile actors \textit{(so that they can conduct malicious actions in restricted areas that require physical access)}.\\

\noindent In order to mitigate these threats, the use of encryption \textit{(provided by RFID/NFC tag architectures and implementations)} can be used to protect information stored on the card and reduce the vulnerability of the tag to cloning and spoofing attacks.\\

\noindent However, it should be noted that it is possible that some organisations, operating large legacy access control systems, may be deploying tags where the encryption system provided was not used. Any personal data stored on those tags \textit{(such as ID numbers)} would then be vulnerable to being stolen by hostile actors, and could possibly be used to track tags \textit{(i.e. people)} in an unauthorised fashion. Such unencrypted tags could also be surreptitiously cloned by hostile actors - sometimes with the relative ease of using an app such as the MIFARE Classic Tool\footnote{\href{https://play.google.com/store/apps/details?id=de.syss.MifareClassicTool&hl=en_GB&rdid=de.syss.MifareClassicTool}{Google Play Store: MIFARE Classic Tool}} which is available on the Google Play Store.\\

\noindent Furthermore, the proprietary encryption scheme deployed with the MIFARE Classic tag family has been compromised since 2008\footnote{\href{http://www.cs.ru.nl/~flaviog/publications/Security_Flaw_in_MIFARE_Classic.pdf}{Schreur, R. et al (2008) \textit{Security Flaw in MIFARE Classic}}} and the hardened MIFARE Classic EV1 tags have been compromised since 2015\footnote{\href{http://cs.ru.nl/~rverdult/Ciphertext-only_Cryptanalysis_on_Hardened_Mifare_Classic_Cards-CCS_2015.pdf}{Meijer, C. and Verdult, R. (2015) \textit{Ciphertext-only Cryptanalysis on Hardened Mifare Classic Cards}}}. These attacks make it possible to clone encrypted MIFARE Classic tags, in a short amount of time, with equipment that is relatively cheap and available to motivated criminal or hostile state actors. The manufacturer, NXP Semiconductors, have officially recommended that MIFARE Classic-based systems are updated to use tags and readers from more secure MIFARE tag families.\footnote{\href{https://www.mifare.net/en/products/chip-card-ics/mifare-classic/frequently-asked-questions/}{MIFARE: FAQs on the security of the MIFARE Classic}} It should be noted, however, that attacks against the MIFARE DESFire\footnote{\href{http://www.proxmark.org/files/Documents/13.56\%20MHz\%20-\%20MIFARE\%20DESFire/Cloning_Cryptographic_RFID_Cards_for_25USD-WISSEC_2010.pdf}{Kasper, T. et al (2016) \textit{Cloning Cryptographic RFID Cards for 25\$}}} and Ultralight\footnote{\href{https://www.youtube.com/watch?v=-uvvVMHnC3c}{YouTube: NFC For Free Rides and Rooms (on your phone)}} families have also been published.\\

\noindent In light of this, it would be prudent to take additional actions to further mitigate the security concerns introduced by RFID ID badge/access control deployments and protect critical organisational assets. The deployment of multiple factor authentication systems \textit{(e.g. a door PIN code and/or biometrics in addition to access cards)} to protect critical assets \textit{(such as server rooms)} could be a robust measure to protect against current and future vulnerabilities in tag implementations. Another mitigative action could be issuing shielded tag carriers or card wallets to users \textit{(to reduce the risk of tag tracking and reads-at-range to steal data from unencrypted or poorly encrypted tags)}.\\

\noindent In conclusion, the current and future risks of RFID access control systems are uncertain. Board members and upper management will have significant stake in decisions taken in the event that a vulnerability is discovered in the access control system deployed in their organisation whether or not the vulnerability is found during a security review/penetration test or as part of a post-incident response analysis. A significant vulnerability in a system could necessitate extensive hardware and infrastructure fixes to rectify. As a result of this, the argument can be made that management should take proactive actions \textit{(such as implementing the mitigative measures detailed above and codifying them in policies)} in order to improve system resilience and guard against future threat. Finally, all RFID ID badge/access control systems should make use of encryption as a baseline.

% \pagebreak
\section{Tangible Demonstration}
Performing a demonstration of these attacks against a live production system would come with numerous legal and ethical issues such as the risk of compromising personal data (if the card tested in the demonstration happens to contain very sensitive personal data) and the risk of violating the terms and conditions surrounding the deployment of the RFID access control system.

Notwithstanding the above, there would also be significant practical issues.
Firstly, the presentation and demonstration could only be performed at venues where permission to test the infrastructure publicly was given, which will not be many. Secondly, some cards (if provided at random by audience members) may be resistant to the variations of attacks vs MIFARE Classic that are being demonstrated. This would result in the demonstration failing; however, the cards may yet be vulnerable to other attacks, such as the DESFire and Ultralight.

With regards to the above, utilising a portable demonstration setup that consists of an access card reader, a cloning device (for example, an Android phone with the MIFARE Classic Tool app installed), an unencrypted MIFARE Classic 4k tag, and a blank UID-writable MIFARE Classic 4k tag would seem to be the optimal solution for providing a consistent demonstration platform. The reader could simply indicate an authentication success (door unlocked) or an authentication failure (door remains locked) with a green and red LED respectively. The process of the demonstration would entail:

\begin{enumerate}
  \item Proving that the card to be cloned to is in fact blank. This would be done by showing that it fails authentication with the access card reader.
  \item Showing that all sectors (some of which could contain example personal data) can be read from unencrypted MIFARE 4k card and dumped to a file on the cloning device (mobile phone) with the MIFARE Classic Tool.
  \item Showing that the dumped image of the original valid card can be cloned to a UID-writable (sector 0 read-write) equivalent tag that implements the MIFARE Classic standard by using the write tag functionality within MIFARE Classic Tool.
  \item Proving that the card cloned to can now be used to authenticate successfully with the access card reader.
\end{enumerate}
