\section{Alternatives}
As LAMP stacks now refer to systems that meet the LAMP stack deployment model, it would be possible to use significantly different alternatives -- these could include using a different Linux server operating system \textit{(such as CentOS)} as the base and a different programming language \textit{(such as Ruby or PHP)} as the web application development language. Significant frameworks \textit{(such as Rails and Laravel)} exist for other languages that provide abstractions \textit{(for validation and other security aspects)} similar to those I have demonstrated for Python3/Flask in this report.

Although it is suggest that \texttt{nginx} is used as a load-balancing proxy for \texttt{gunicorn}, it is also possible to use Apache \textit{(with some configuration tweaks)} in the place of \texttt{nginx}. Furthermore, \texttt{MySQL} is one of many database systems -- others may have features that make them more attractive for specific use-cases \textit{(such as high performance distributed database clusters)} over \texttt{MySQL}.

Regardless, it would make sense to at least attempt to use Docker with any alternative combination of the above. Docker images for many different database systems \textit{(such as MySQL, PostgreSQL)} and servers \textit{(e.g. Apache, nginx)} are a mature part of the Docker ecosystem. The use of containers helps to cleanly separate the components of the system from each other in a way that improves security and scalability.

In a production environment, for a significant application infrastructure where consistent availability is a must, consideration should be given towards deploying container orchestration technologies such as Kubernetes. In this architecture, which is really just a distributed LAMP stack, Kubernetes is used to connect multiple servers together in a cluster. Containers, in the form of pods, are then deployed automatically by the Kubernetes orchestrator to keep the system in line with the system definitions defined.

With such an architecture, to obtain maximal resilience and availability, Docker runtime environments could be deployed on multiple distributions of Linux and across multiple cloud providers \textit{(such as AWS, Azure, etc)} to provide a stable base for the containerised LAMP stack whilst also reducing the risk introduced by the discovery of a vulnerability in any specific distribution.
