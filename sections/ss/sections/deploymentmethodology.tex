\section{Deployment Methodology}
Underlined is a basic methodology for deploying the server, MySQL, and the web application with gunicorn.

\begin{enumerate}
  \item Set up Ubuntu Server 18.10 \footnote{\href{http://releases.ubuntu.com/18.10/ubuntu-18.10-live-server-amd64.iso}{\texttt{ubuntu-18.10-live-server-amd64.iso}}} using the Subiquity\footnote{\href{https://github.com/CanonicalLtd/subiquity}{\faGithub\ CanonicalLtd/subiquity}} installer
    \begin{enumerate}[label=\roman*\textasciitilde]
      \item Select \textit{Install Ubuntu Server} at the GRUB2 boot menu\\
      \textcolor{nicer-blue}{[Figure~\ref{fig:IY2D502-2019-02-21-19-15-19}]}
      \item For standard UK keyboards, set the layout and variant to \textit{English {UK}} and \textit{English (UK, extended, with Win keys)} respectively -- this will help to ensure that passwords are entered correctly\\
      \textcolor{nicer-blue}{[Figure~\ref{fig:IY2D502-2019-02-21-19-16-03}]}
      \item Select \textit{Install Ubuntu} for a normal server deployment \todo{talk a bit about MAAS alternative here}\\
      \textcolor{nicer-blue}{[Figure~\ref{fig:IY2D502-2019-02-21-19-16-06}]}
      \item Choose to set up Logical Volume Management \textit{(LVM)} when prompted to configure filesystem setup\\
      \textcolor{nicer-blue}{[Figure~\ref{fig:IY2D502-2019-02-21-19-16-38}]}
      \item Confirm the default LVM filesystem configuration provided by the installer \todo{note how logical volumes could be adjusted to have separate /tmp, /var for logs, etc partitions (with nodev,noexec,etc) for improved security [IY2D502-2019-02-21-19-16-43.png]}\\
      \textcolor{nicer-blue}{[\textit{Figure~\ref{fig:IY2D502-2019-02-21-19-16-43}}]}
      \item Configure the server by providing a name, server name \textit{(hostname)}, username, password -- it is also possible to import SSH identities from GitHub, as exhibited in the screenshot, and thus configure immediate secure access over SSH with public-private key authentication\\
      \textcolor{nicer-blue}{[Figure~\ref{fig:IY2D502-2019-02-21-19-17-27}]}
      \item Reboot when prompted to finish the installation
      \item Upon logging into the new server install, check for updates and upgrade if required -- on Ubuntu, this is done with \texttt{apt} using the commands:
        \begin{itemize}
          \item \term{sudo apt update}
          \item \term{sudo apt upgrade}
        \end{itemize}
      \textcolor{nicer-blue}{[Figure~\ref{fig:IY2D502-2019-02-21-19-22-44}]}
    \end{enumerate}
  \item Download latest \texttt{salapp}
    \begin{enumerate}[label=\roman*\textasciitilde]
      \item Clone \texttt{salapp} from \href{https://github.com/david-wm-sanders/uswacs-2-iy2d502-salapp}{\faGithub\ david-wm-sanders/uswacs-2-iy2d502-salapp} using:\\
      \term{git clone git@github.com:david-wm-sanders/uswacs-2-iy2d502-salapp.git}
    \end{enumerate}
  \item Install \texttt{docker-ce}\footnote{\href{https://github.com/docker/docker-ce}{\faGithub\ docker/docker-ce}} \textit{(Docker Community Edition)}
    \begin{enumerate}[label=\roman*\textasciitilde]
      \item Use \term{sudo apt install docker.io} to install \texttt{docker} on the server\\
      \textcolor{nicer-blue}{[Figure~\ref{fig:IY2D502-2019-02-21-23-29-41}]}
    \end{enumerate}
  \item Set up a containerised MySQL instance using docker
    \begin{enumerate}[label=\roman*\textasciitilde]
      \item run \term{./start\_mysql.sh} \todo{include listing here}
    \end{enumerate}
  \item Deploy salapp with docker
    \begin{enumerate}[label=\roman*\textasciitilde~]
      \item build container with \term{./make\_salapp.sh} \todo{include listings here}
      \item start salapp with \term{./start\_salapp.sh} \todo{include listings here}
    \end{enumerate}
  \item \todo{Connect to web application and trigger shit}
  \item \todo{Show how \texttt{mysql\_root\_login.sh} and \texttt{mysql\_salapp\_login.sh} can be used to check that the quotes were entered into the db correctly}
  \item \todo{Create a script to view the security alert logs and demonstrate with evidence}
\end{enumerate}
