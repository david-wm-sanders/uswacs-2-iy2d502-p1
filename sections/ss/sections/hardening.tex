\section{Hardening}
\todo{Hardening intro...}

% \pagebreak
\subsection{Ubuntu 18.10 Server: \texttt{salsrv}}
\todo{some text here pls}
% \subsubsection*{Logical Volume Management}
% \paragraph{Partitioning}
% As discussed in the methodology, and although beyond the scope of this report, it would be possible to use Logical Volume Management to define a more resilient partitioning scheme on disk.
%
% This could include having separate paritions for \texttt{/tmp} and \texttt{/var}. In the case of \texttt{/tmp}, benefit would be gained by mounting the drive with \texttt{noexec} if the server was storing user uploaded data in the \texttt{/tmp} partition.
%
% Having separate partitions also increases the resilience of the system by preventing logs, written to \texttt{/var/log} for example, from filing the drive completely and disrupting the operation of the \texttt{/} partition. An attacker could attempt to trigger events that cause logs to be written in order to fill server drives and cause a Denial-of-Service.
%
% In light of this, the use of more finessed partitioning schemes can help to ensure that the integrity and availability of data and the server itself is maintained.
%
% \paragraph{Encryption}
% Fully encrypting drives with LUKS would improve confidentiality significantly
% \subsubsection{User Accounts}
% Locking them down more?
\subsubsection{Locking down SSH via \texttt{/etc/ssh/sshd\_config}}
% https://medium.com/@jasonrigden/hardening-ssh-1bcb99cd4cef
% https://linux-audit.com/audit-and-harden-your-ssh-configuration/
% https://www.cyberciti.biz/tips/linux-unix-bsd-openssh-server-best-practices.html
% https://github.com/BetterCrypto/Applied-Crypto-Hardening/blob/master/src/configuration/SSH/OpenSSH/6.6/sshd_config
% https://gist.github.com/tribou/fcda8e6066776c9eaa47 (with added TORHS goodies...)
% https://security.stackexchange.com/questions/179114/what-are-the-toughest-ssh-daemon-settings-in-terms-of-encryption-handshake-or
% https://security.stackexchange.com/questions/154076/hardening-ssh-security-on-a-debian-9-server
% https://infosec.mozilla.org/guidelines/openssh (taking it to the next level!)

In order to prevent unauthorised remote access via SSH to the server, several steps should be taken to improve the security of the SSH server configuration. This will increase the confidentiality of data on the server \textit{(by further restricting access to the server)} whilst also maintaining integrity and availability for authenticated users.

An element of improving the SSH configuration was performed during the installation by importing SSH identities from GitHub. This allows the user who controls the private keys of those identities to connect to the server out-of-the-gate with public-private key authentication rather than password authentication for SSH connections.

Further harden the SSH server by modifying the configuration in \texttt{/etc/ssh/sshd\_config} \textit{(as shown in Figure~\ref{fig:IY2D502-2019-02-26-17-44-30})}. This involves:
\begin{itemize}
  \item Preventing remote login over SSH to the \texttt{root} account by setting:\\
    \term{PermitRootLogin no}
  \item Setting \term{PasswordAuthentication no} to prevent password authentication for all incoming SSH connections to all accounts completely -- this can be done without worry if the \texttt{authorized\_keys} imported from GitHub during the installation work successfully
\end{itemize}

\subsubsection{Setting up a Firewall}
A firewall can be used to drop all traffic to ports not explicitly allowed to receive incoming traffic -- this effectively closes the port to the outside world.

On Ubuntu, firewall configuration can be performed with the \texttt{ufw} tool \textit{(as shown in Figure~\ref{fig:IY2D502-2019-02-26-19-28-12})}.

This involves:
\begin{itemize}
  \item Starting the firewall with \term{sudo ufw enable}
  \item Adding allow rules with:
    \begin{itemize}
      \item \term{sudo ufw allow ssh}: will allow inbound traffic on port 22
      \item \term{sudo ufw allow http}: will allow inbound traffic on port 80
      \item \term{sudo ufw allow https}: will allow inbound traffic on port 443
    \end{itemize}
  \item Checking that the firewall is active and configured correctly using:\\
    \term{sudo ufw status}
\end{itemize}

\subsubsection{Ensuring that the system in upgraded}
Ensure that the server is patched regularly \textit{(as shown in Figures~\ref{fig:IY2D502-2019-02-22-23-41-29}~and~\ref{fig:IY2D502-2019-02-22-23-44-52})}. Security vulnerabilities in system packages could expose the server to exploits that compromise confidentiality, integrity, and/or availability.

During upgrading it is possible to ascertain the changes \textit{(some of which will be fixes for security vulnerabilities)} made to the packages. For example, in Figure~\ref{fig:IY2D502-2019-02-22-23-44-52}, we can see that \texttt{bind9} was upgraded to version \texttt{1:9.11.4+dfsg-3ubuntu5.1}. Changelogs for Ubuntu packages are published on Launchpad and the changelog for \texttt{bind9} at this version at:\\
\href{https://launchpad.net/ubuntu/+source/bind9/1:9.11.4+dfsg-3ubuntu5.1}{\texttt{https://launchpad.net/ubuntu/+source/<package>/<version}}.

In this changelog, we can see that several security fixes were made by patching the source code for \texttt{bind9}.

In a production environment, a crontab job could be developed to check for updates daily and alert the server administrator(s) when there are security updates.
% \subsubsection*{Anti-Malware: \texttt{rkhunter?}}

% \pagebreak
\subsection{Flask application: \texttt{salapp}}
In developing my web application, I endeavoured to embed security and hardening from the start.

I chose to use Python3 as the development language due to my existing familiarity with the language and Flask as the framework because it is lightweight and mature with a significant ecosystem of existing Flask extensions that enhance security by adding CSRF to forms, allowing the definition of custom validators for form inputs, setting security headers, enforcing HTTPS, and so on.

Although I had not used Flask before, I was able to pick it up relatively quickly and integrate security considerations into the web application design from the beginning.

\subsubsection{SQL Injection}
The \texttt{Flask\_SQLAlchemy} extension is used to use to provide a link between the \texttt{SQLAlchemy} Python library and the Flask web application. If the Object Relational Mapper, provided by \texttt{SQLAlchemy}, is used correctly it is possible to correctly escape all inputs for the database and prevent SQL Injection attacks -- in short, this entails always use the ORM primitives and never directly formulating a SQL query.

The \texttt{quotes} table, and columns therein, are defined in Python code in \texttt{salapp/app/models.py} \textit{(shown in Listing~\ref{lst:hard:models.py})}. It is possible here to set constraints on fields in the table, such as the length for the \texttt{db.String} type.

This model definition will be used by \texttt{SQLAlchemy} to interact with the database and can even be used to create the SQL query statements to create new tables that store the fields defined in the model in a database.
\begin{listing}[H]
  \captionsetup{skip=\skiplistingcaptionlen}
  \inputminted[breakanywhere]{python3}{../uswacs-2-iy2d502-salapp/app/models.py}
  \caption{\texttt{salapp/app/models.py}}
  \label{lst:hard:models.py}
\end{listing}

\subsubsection{User Input Validation}
When taking data from clients via the quotes form, it is important to validate the input to ensure that it meets requirements and does not exceed constraints before inserting it in the database. The form definition and validation logic is handled in \hyperref[fcl:uswacs-2-iy2d502-salapp:forms.py]{salapp/app/forms.py}.
