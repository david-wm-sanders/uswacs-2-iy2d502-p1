\section{Introduction}
\todo{an overarching view of the project requirements, methodology adopted, report layout, tests done, secure development strategies used, etc}

For this assignment, my task was to deploy and secure a LAMP stack. As a solution for deploying web applications and sites, the term `LAMP' originally comes from the components: Linux \textit{(as the deployment operating system)}, Apache \textit{(as the HTTP web server)}, MySQL \textit{(as the database server/management system)}, and PHP \textit{(as the web site/application development language)}.

However, in modern usage, the term "LAMP" refers to a deployment model where alternatives can be used in lieu of the original components.

In seeking to meet the security constraints of the requirements, I endeavoured to choose development and deployment frameworks that would enable me to secure the web application and server as effectively as possible from the start.

As such, I decided to deploy the MySQL and gunicorn/Flask instances in separate docker containers.

\todo{more fluff here about docker, secure development with flask/python - it's easy af, testing strategy?}

\begin{displaytable}{\label{display:lamp_stack}}{Dockerised LAMP Stack}
  \paragraph{\href{http://releases.ubuntu.com/18.10/ubuntu-18.10-live-server-amd64.iso}{\faUbuntu} Ubuntu Server 18.10}
  \hspace{-0.6em}-- as the deployment operating system.
  \paragraph{\href{https://hub.docker.com/\_/mysql}{\faDatabase} MySQL}
  \hspace{-0.6em}-- as the database server/management system, running in a docker container \todo{and only exposed to other containers explicitly?}
  \paragraph{\href{https://pypi.org/project/gunicorn/}{\faServer} Gunicorn}
  \hspace{-0.6em}-- as the WSGI HTTP Server, running the web application in a custom built container based on the python3.7-slim docker image.
  \paragraph{\href{https://pypi.org/project/Flask/1.0.2/}{\faPython} Flask}
  \hspace{-0.6em}-- as the web application development framework, with Python3 and jinja as the development languages.
  \vspace{1em}
\end{displaytable}
