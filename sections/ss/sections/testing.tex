\section{Security Testing}
% \todo{intro 100 words}
\subsection{Use Nmap to check firewall}
\subsubsection*{Plan}
Use \texttt{nmap} in order to perform a port scan of the server in order to verify that the firewall is operating correctly.
\begin{itemize}
  \item Run \term{sudo nmap -p- <server\_ip>} to perform a full SYN scan of all ports at the target.
\end{itemize}

\subsubsection*{Test Results Summary}
As shown in Listing~\ref{lst:test:nmap_results}, all ports other than 22, 80, and 443 are being filtered by the firewall that was configured during server hardening in Section~\ref{sec:hardening:firewall}.
\begin{listing}[H]
  \captionsetup{skip=\skiplistingcaptionlen}
  \begin{minted}[breakanywhere]{bash}
    dwms@attack:~$ sudo nmap -p- 192.168.79.133

    Starting Nmap 7.60 ( https://nmap.org ) at 2019-02-26 20:38 UTC
    Stats: 0:00:01 elapsed; 0 hosts completed (1 up), 1 undergoing SYN Stealth Scan
    SYN Stealth Scan Timing: About 0.01% done
    ...
    Stats: 2:35:48 elapsed; 0 hosts completed (1 up), 1 undergoing SYN Stealth Scan
    SYN Stealth Scan Timing: About 90.64% done; ETC: 23:30 (0:16:05 remaining)
    Nmap scan report for 192.168.79.133
    Host is up (0.0010s latency).
    Not shown: 65532 filtered ports
    PORT    STATE SERVICE
    22/tcp  open  ssh
    80/tcp  open  http
    443/tcp open  https
    MAC Address: 00:0C:29:49:42:5A (VMware)

    Nmap done: 1 IP address (1 host up) scanned in 10405.47 seconds
  \end{minted}
  \caption{Nmap Scan Results}
  \label{lst:test:nmap_results}
\end{listing}

\pagebreak
\subsection{Use Nessus to check web application}
\subsubsection*{Plan}
Use Nessus to test the server for web application vulnerabilities.
\begin{itemize}
  \item Set up a Nessus instance
  \item Create a new Web Application scan with the target set to the server
  \item Review and export the test results
  \item Plan further hardening tasks
\end{itemize}
\subsubsection*{Test Results Summary}
The test reported 1 medium vulnerability relating to click-jacking because the server does not set \texttt{X-Frame-Options} or \texttt{Content-Security-Policy frame-ancestors} headers. The Executive Summary of the report is included in Appendix~\ref{app:ntr:salapp_test_2}.

It should be noted that no vulnerabilities related to SQL injection, user input validation, CSRF, or HTTPS were found in this scan.
\subsubsection*{Further Hardening}
\begin{itemize}
  \item Integrate \texttt{Flask-Talisman} into the \texttt{salapp} web application. The \texttt{Flask-Talisman} extension can be used to handle the task of setting HTTP security headers.
\end{itemize}

\noindent\textcolor{deep-gray}{Due to time constraints, implementing \texttt{Flask-Talisman} in \texttt{salapp} was deemed beyond project scope.}

% \pagebreak
% \subsection{Lynis}
% \subsubsection*{Plan}
% \todo{50w}
% \subsubsection*{Test Results Summary}
% \todo{50w}
%
%
%
% \todo{200w}
